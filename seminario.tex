\documentclass{beamer}

\usetheme{Copenhagen}
\setbeamercolor{structure}{fg=red!70!black}
\setbeamertemplate{navigation symbols}{}
\setbeamertemplate{headline}{}
\usepackage{transparent}
\logo{\transparent{0.03}\includegraphics[width=1.3\textwidth]{img/unimi_logo.pdf}}

\usepackage{settings-seminario}

\title{Automi 2-limited e linguaggi context-free}
\author{Alessandro Clerici Lorenzini}
\date{luglio 2023}
\institute{Seminario per l'esame di Teoria dei Linguaggi @ Unimi}

\begin{document}
\maketitle


\section{Introduzione}


\section{Automi \texorpdfstring{$2$}{2}-limited}
\begin{frame}{Automi $d$-limited}
	\begin{itemize}
		\item Hibbard (1967), caratterizzazione del determinismo nei linguaggi liberi dal contesto
		\item $d$-limited (\la d): restrizione di una macchina di Turing e al contempo generalizzazione degli automi \emph{two-way}:
		      \begin{itemize}
			      \item spazio di lavoro limitato alle celle contenenti l'input tramite \emph{end-marker}
			      \item scrittura su ciascuna cella solo durante le sue prime $d$ visite
		      \end{itemize}
	\end{itemize}

	\vfill
	\begin{figure}
		\centering
		\begin{tikzpicture}
	\node[cell] (0) {$\sigma_0$};
	\node[cell] (1) [right=of 0] {$\sigma_1$};
	\node[cell] (2) [right=of 1] {$\sigma_2$};
	\node[cell] (3) [right=of 2] {$\sigma_3$};
	\node[cell,minimum width=40pt] (rd) [right=of 3]{$\cdots$};
	\node[cell] (last) [right=of rd] {$\sigma_n$};
	\node[node distance=0pt] (rem) [right=of last]{\Large $\rem$};
	\node[node distance=0pt] (lem) [left=of 0]{\Large $\lem$};

	\node (q) [below=.5 cm of 2] {$q$};
	\draw[-latex,shorten >=1pt] (q) -- (2);
\end{tikzpicture}

	\end{figure}
\end{frame}

\begin{frame}{Automi $2$-limited: un esempio}
	\framesubtitle{Linguaggi di Dyck}
\end{frame}

\begin{frame}{Potenza riconoscitiva: tutti i linguaggi context-free}
\end{frame}

\begin{frame}{Potenza e complessità: altre facce della medaglia}
\end{frame}

\section{Da \texorpdfstring{$2$}{2}-limited a PDA}
\begin{frame}{{Da \texorpdfstring{$2$}{2}-limited a PDA}}
\end{frame}


\section{Altri aspetti}
\begin{frame}{Da PDA a \texorpdfstring{$2$}{2}-limited: complessità}
\end{frame}

\begin{frame}{Problemi aperti e ricerche future}
\end{frame}

\end{document}
