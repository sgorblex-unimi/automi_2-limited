\documentclass{beamer}

\usetheme{Copenhagen}
\setbeamercolor{structure}{fg=red!75!black}
\setbeamertemplate{navigation symbols}{}
\setbeamertemplate{headline}{}
\usepackage{transparent}
\logo{\transparent{0.03}\includegraphics[width=1.3\textwidth]{img/unimi_logo.pdf}}

\usepackage{settings-seminario}

\title{Automi 2-limited e linguaggi context-free}
\author{Alessandro Clerici Lorenzini}
\date{luglio 2023}
\institute{Seminario per l'esame di Teoria dei Linguaggi @ Unimi}

\begin{document}
\maketitle


\section{Introduzione}


\section{Automi \texorpdfstring{$2$}{2}-limited}
\begin{frame}{Automi $d$-limited}
	\begin{itemize}
		\item Hibbard (1967), caratterizzazione del determinismo nei linguaggi liberi dal contesto
		\item $d$-limited (\la d): restrizione di una macchina di Turing e al contempo generalizzazione degli automi \emph{two-way}:
		      \begin{itemize}
			      \item spazio di lavoro limitato alle celle contenenti l'input tramite \emph{end-marker}
			      \item scrittura su ciascuna cella solo durante le sue prime $d$ visite
		      \end{itemize}
	\end{itemize}

	\vfill
	\begin{figure}
		\centering
		\begin{tikzpicture}
	\node[cell] (0) {$\sigma_0$};
	\node[cell] (1) [right=of 0] {$\sigma_1$};
	\node[cell] (2) [right=of 1] {$\sigma_2$};
	\node[cell] (3) [right=of 2] {$\sigma_3$};
	\node[cell,minimum width=40pt] (rd) [right=of 3]{$\cdots$};
	\node[cell] (last) [right=of rd] {$\sigma_n$};
	\node[node distance=0pt] (rem) [right=of last]{\Large $\rem$};
	\node[node distance=0pt] (lem) [left=of 0]{\Large $\lem$};

	\node (q) [below=.5 cm of 2] {$q$};
	\draw[-latex,shorten >=1pt] (q) -- (2);
\end{tikzpicture}

	\end{figure}
\end{frame}

\begin{frame}{Automi $2$-limited: un esempio}
	\framesubtitle{Linguaggi di Dyck}
	Dato $k\in\N^+$
	\begin{itemize}
		\item $\Omega_k$ è l'alfabeto composto da $k$ tipi di parentesi ($2k$ simboli in tutto)
		\item $D_k$ è il linguaggio delle parentesi bilanciate su $\Omega_k$, detto linguaggio di Dyck.
	\end{itemize}
	\vspace{1cm}
	Mostriamo come un \la2 può riconoscere $D_k$.
\end{frame}

\begin{frame}{Potenza riconoscitiva: tutti i linguaggi context-free}
	\begin{theor}[di Chomsky-Schützenberger]
		Se $L$ è un linguaggio context-free su alfabeto $\Sigma$, esistono un naturale $k\in0$, un linguaggio regolare $R$ e un omomorfismo $h:\Omega_k\to\sigma\star$ tali che
		\begin{equation*}
			L=h(D_k\cap R)
		\end{equation*}
	\end{theor}
	\only<2>{
		\vfill
		Un \la2 può simulare il comportamento di una composizione di macchine:
		\begin{itemize}
			\item una macchina one-way che su input $w$ produce nondeterministicamente una stringa $z\in h^{-1}(w)$;
			\item il \la2 presentato precedentemente per $D_k$;
			\item un automa a stati finiti per $R$.
		\end{itemize}
	}
\end{frame}

\begin{frame}{Potenza e complessità: altre facce della medaglia}
	\begin{itemize}
		\item i \la2 riconoscono solo context-free? I.e., si possono convertire in PDA?
		\item qual è il rapporto di complessità tra le diverse descrizioni (PDA, \la2)?
		\item cosa succede nel caso deterministico?
	\end{itemize}
\end{frame}

\section{Da \texorpdfstring{$2$}{2}-limited a PDA}
\begin{frame}{Da \texorpdfstring{$2$}{2}-limited a PDA}
	\only<1>{
		\framesubtitle{Tabelle di transizione}
		\begin{columns}[t]
			\column{0.5\textwidth}
			\begin{figure}[h]
				\centering
				\begin{tikzpicture}[tapeseg/.style={minimum height=1.2em,minimum width=1.5em,outer sep=0pt,node distance=0pt}]
	\footnotesize
	\node[tapeseg] (w) {$w$};
	\draw	([xshift=-2cm]w.north) -- ([xshift=+2cm]w.north)
		([xshift=-2cm]w.south) -- ([xshift=+2cm]w.south);
	\node at ([yshift=-0.3cm, xshift=-2.3cm]w.south) (q) {$q$};
	\node (p) [below=.2 of q] {$p$};
	\draw	(q.east) -- ++(+2cm,-.12cm)
		-- ++(-.9cm,-.09cm) -- ++(+1.8cm,-.10cm)
		-- ++(-2.7cm,-.12cm) -- ++(+1.55cm,-.10cm)
		-- (p.east);
\end{tikzpicture}

				\caption{$(q,-1,p,-1)\in\tau_w$}
			\end{figure}
			\begin{figure}[h]
				\centering
				\begin{tikzpicture}
	\footnotesize
	\node[tapeseg] (w) {$w$};
	\draw	([xshift=-2cm]w.north) -- ([xshift=+2cm]w.north)
		([xshift=-2cm]w.south) -- ([xshift=+2cm]w.south);
	\node at ([yshift=-0.3cm, xshift=+2.3cm]w.south)	(q) {$q$};
	\node at ([yshift=-6mm, xshift=-2.3cm]w.south)		(p) {$p$};
	\draw	(q.west) -- ++(-3.5cm,-.12cm)
		-- ++(+2.5cm,-.09cm) -- (p.east);
\end{tikzpicture}

				\caption{$(q,+1,p,-1)\in\tau_w$}
			\end{figure}
			\column{0.5\textwidth}
			\begin{figure}[h]
				\centering
				\begin{tikzpicture}
	\footnotesize
	\node[tapeseg] (w) {$w$};
	\draw	([xshift=-2cm]w.north) -- ([xshift=+2cm]w.north)
		([xshift=-2cm]w.south) -- ([xshift=+2cm]w.south);
	\node at ([yshift=-0.3cm, xshift=-2.3cm]w.south)	(q) {$q$};
	\node at ([yshift=-8mm, xshift=2.3cm]w.south)		(p) {$p$};
	\draw	(q.east) -- ++(+3.5cm,-.12cm)
		-- ++(-2.5cm,-.09cm) -- ++(+2cm,-.10cm)
		-- ++(-2.7cm,-.12cm) -- (p.west);
\end{tikzpicture}

				\caption{$(q,-1,p,+1)\in\tau_w$}
			\end{figure}
			\begin{figure}[h]
				\centering
				\input{img/transtable++.tikz}
				\caption{$(q,+1,p,+1)\in\tau_w$}
			\end{figure}
		\end{columns}
		\vfill
		Composizione: $\tau_w\cdot\tau_z:=\tau_{wz}$
	}
	\only<2>{
		\framesubtitle{Tabelle di uscita}
		\begin{columns}[t]
			\column{0.5\textwidth}
			\begin{figure}[h]
				\centering
				\input{img/exittable--.tikz}
				\caption{$(p,-1)\in\exitt(\tau_w,q,-1,\tau_z)$}
			\end{figure}
			\begin{figure}[h]
				\centering
				\input{img/exittable-+.tikz}
				\caption{$(p,-1)\in\exitt(\tau_w,q,+1,\tau_z)$}
			\end{figure}
			\column{0.5\textwidth}
			\begin{figure}[h]
				\centering
				\begin{tikzpicture}[tapeseg/.style={minimum height=1.1em,minimum width=1.5em,outer sep=0pt,node distance=0pt}]
	\footnotesize
	\node[tapeseg] (center) {};
	\node[tapeseg,node distance=.5] (w) [left=of center] {$w$};
	\node[tapeseg,node distance=.5] (z) [right=of center] {$z$};
	\draw	([xshift=-2cm]center.north) -- ([xshift=+2cm]center.north)
	([xshift=-2cm]center.south) -- ([xshift=+2cm]center.south)
	(center.north) -- (center.south);
	\node (q) at ([xshift=2mm,yshift=-3mm]center.south) {$q$};
	\node (p) at ([xshift=2.2cm,yshift=-7mm]center.south) {$p$};
	\draw	(q.west) -- ++(-1cm,-.12cm)
	-- ++(+2.5cm,-.09cm) -- ++(-3cm,-.09cm)
	-- (p.west);
\end{tikzpicture}

				\caption{$(p,+1)\in\exitt(\tau_w,q,-1,\tau_z)$}
			\end{figure}
			\begin{figure}[h]
				\centering
				\input{img/exittable++.tikz}
				\caption{$(p,+1)\in\exitt(\tau_w,q,+1,\tau_z)$}
			\end{figure}
		\end{columns}
	}
\end{frame}
\begin{frame}{PDA per $L\rem$}
	\only<1>{
		\framesubtitle{Definizioni}
		$L\in\mathrm{CFL}$

		$M=(Q,\Sigma,\Gamma,\delta,q_0,F)$ tale che $\genlang(M)=L$

		Costruiamo $M'=(Q',\Sigma,\Gamma',\delta',q_0',Z_0',F')$ tale che $\genlang(M')=L\rem$

		\begin{gather*}
			Q':=Q\cup Q\times\set{\tau_w\mid w\in\Sigma\star} \\
			\Gamma':=\Gamma\cup\set{\tau_w\mid w\in\Sigma\star}
		\end{gather*}

		\begin{itemize}
			\item Mantiene sulla pila lo stato del nastro di $M$ fino al simbolo letto, usando simboli per celle riscrivibili e tabelle per i segmenti congelati.
			\item In \emph{normal mode} legge il prossimo simbolo e carica sulla pila quanto scritto da $M$, in \emph{back mode} simula il comportamento di $M$ in una computazione nella parte del nastro già visitata, agendo per $\emptyword$-mosse sulla pila.
		\end{itemize}
	}
	\only<2>{
		\framesubtitle{Normal mode}
		In normal mode viene simulato direttamente il comportamento di $M$
	}
	\only<3>{
		\framesubtitle{Back mode: 1 - entrata}
		In back mode lo stato è una coppia $(q,T)$ composta dallo stato simulato $q$ di $M$ e la tabella di transizione $T$ dell'ultimo segmento congelato (i.e. quello che termina con l'ultima cella letta).
	}
	\only<4>{
		\framesubtitle{Back mode: 2 - simbolo da riscrivere}
	}
	\only<5>{
		\framesubtitle{Back mode: 3 - segmento congelato}
	}
	\only<6>{
		\framesubtitle{Back mode: 4 - uscita}
	}
\end{frame}


\section{Altri aspetti}
\begin{frame}{Da PDA a \texorpdfstring{$2$}{2}-limited: complessità}
\end{frame}

\begin{frame}{Problemi aperti e ricerche future}
\end{frame}

\end{document}
